\documentclass[10pt]{article}

% \usepackage{underscore}
\makeatletter
\catcode`\_\active
\DeclareRobustCommand{\myunderscore}{\ifmmode\sb\else
  \leavevmode\nobreak\hskip\z@skip 
  \_\-\nobreak\hskip\z@skip \fi}
\let_\myunderscore
\makeatother


\usepackage{geometry}
\geometry{left=1in, right=1in, top=1in, bottom=1in}

\usepackage{amsmath}

\setlength{\parindent}{0pt}
\setlength{\parskip}{6pt plus 1pt minus 1pt}

\title{Dependent Events / kickStart-2021-round-H}
\author{Yotam Medini}

\begin{document}

\maketitle

See \texttt{codingcompetitions.withgoogle.com/kickstart/round/0000000000435914/00000000008d9970}

Let \((e_i)_{i=0}^n\) be a chain of dependent event.
Denote \(p_i = P(e_i)\) the probabiliy of \(e_i\) occurs.
In General for \(i > 0\)
\begin{equation*}
p_i 
  = a_i\cdot p_{i-1} + b_i\cdot (1 - p_{i-1})
  = (a_i - b_i)\cdot p_{i-1} + b_i.
\end{equation*}
We denote \(d_i = a_i - b_i\) and  \(D_b^e = \prod_{j=b}^e d_j\).
Now
\begin{equation*}
p_i = d_i\cdot p_{i-1} + b_i % No No = (-d_i)\cdot p_{i-1} + a_i
= p_{i-1}(d_i\cdot 1 + b_i) + (1 - p_{i-1})(d_i\cdot 0 + b_i)
\end{equation*}

Assuming \(e_i\) then \(p_i = 1\) and
\begin{align*}
p_{i+1} &= a_{i+1} p_i + b_{i+1}(1 - p_o) = a_{i+1} \\
p_{i+2} &= d_{i+2}p_{i+1} + b_{i+2} \\
       &= d_{i+2}a_{i+1} + b_{i+2} \\
p_{i+3} &= d_{i+3}p_{i+2} + b_{i+3} \\
       &= d_{i+3}d_{i+2}a_{i+1} + d_{i+3}b_{i+2} + b_{i+3} \\
\cdots &= \cdots \\
p_{i+k} &= \left(a_{i+1}\prod_{j=2}^{k+1}d_{i+j}\right)
       +  \sum_{s=2}^{k+1} b_{i+s} \prod_{j=i+s+1}^{i+k} d_j .
\end{align*}


Assuming \emph{not} \(e_i\) then \(p_i = 0\) and
\begin{align*}
p_{i+1} &= a_{i+1} p_i + b_{i+1}(1 - p_o) = b_{i+1} \\
p_{i+2} &= d_{i+2}p_{i+1} + b_{i+2} \\
       &= d_{i+2}b_{i+1} + b_{i+2} \\
p_{i+3} &= d_{i+3}p_{i+2} + b_{i+3} \\
       &= d_{i+3}d_{i+2}b_{i+1} + d_{i+3}b_{i+2} + b_{i+3} \\
\cdots &= \cdots \\
p_{i+k} &= \sum_{s=1}^{k+1} b_{i+s} \prod_{j=i+s+1}^{i+k} d_j .
\end{align*}


Assuming nothing anout \(e_i\)
\begin{align*}
p_{i+1} &= d_{i+1}p_i + b_{i+1}\\
p_{i+2} &= d_{i+2}p_{i+1} + b_{i+2} \\
        &= d_{i+2} d_{i+1} p_i + d_{i+2} b_{i+1} + b_{i+2} \\
p_{i+3} &= d_{i+3}p_{i+2} + b_{i+3} \\
       &= d_{i+3}d_{i+2}d_{i+1}p_i + d_{i+3}d_{i+2}b_{i+1} + 
          d_{i+3}b_{i+2} + b_{i+3} \\
\cdots &= \cdots \\
p_{i+k} &= p_i \prod_{j=i+1}^{i+k} d_j +
          \sum_{s=1}^k b_{i+s} \prod_{j=i+s+1}^{i+k} d_j \\
       &= \sum_{s=0}^{i+k} (p_i[s=0] + b_{i+s}[s>0]) \prod_{j=i+s+1}^{i+k} d_j \\
\end{align*}

The above expression uses \emph{Iverson brackets},
and when the \(\prod\) runs on empty range, its value
is considered as $1$.
 
Since the above right-side expressions depend on \(p_i\)
we denote, for example, the latter as \(P_{i \to i+k}\)
and specifically \(P_{i+k} = P_{0 \to i+k}\).
Our goal is to have simple formula for \(P_{i\to i+k}\)
depending on \(P_i\) and \(P_{i+k}\).


\begin{align*}
P_{i+k} - P_i \cdot D_{i+1}^{i+k} 
=& \left(\sum_{s=0}^{i+k} (p_0[s=0] + b_s[s>0]) \prod_{j=s+1}^{i+k} d_j\right) + \\
 & D_{i+1}^{i+k} 
   \left(
   \sum_{s=0}^i (p_0[s=0] + b_s[s>0]) \prod_{j=s+1}^i d_j
   \right) \\
=& \left(\sum_{s=0}^{i+k} (p_0[s=0] + b_s[s>0]) \prod_{j=s+1}^{i+k} d_j\right) + \\
 & \left(
   \sum_{s=0}^i (p_0[s=0] + b_s[s>0]) \prod_{j=s+1}^{i+k} d_j
   \right) \\
=& \sum_{s=i+1}^{i+k} (p_0[s=0] + b_s[s>0]) \prod_{j=s+1}^{i+k} d_j \\
=& \sum_{s=i+1}^{i+k} b_s \prod_{j=s+1}^{i+k} d_j 
    = \sum_{s=1}^k b_{i+s} \prod_{j=i+s+1}^{i+k} d_j \\
=& P_{i\to i+k} - p_i D_{1+1}^{i+k}.
\end{align*}

Hence 
\begin{equation*}
P_{i\to i+k} 
  = P_{i+k} - P_i \cdot D_{i+1}^{i+k} + p_i  D_{i+1}^k
  = P_{i+k} + (p_i - P_i) \cdot D_{i+1}^{i+k} 
\end{equation*}

Or more precisely -- assuming or \emph{not} assuming \(e_i\):

\begin{equation*}
P_{i\to i+k} = P_{i+k} + (1\cdot[e_i] - P_i) \cdot D_{i+1}^{i+k} 
\end{equation*}


\end{document}

